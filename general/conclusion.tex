This thesis explored the use of deep learning for the segmentation of droplets in images of fluids vaporized using surface acoustic waves.
The overall goal was to develop a model that can be used to detect the presence of droplets in such images and to employ it in a usable system that measures the size of the droplets.

It gave an overview to the physical background of the atomization techniques and mentioned some of the limitations of other measurement techniques.
Furthermore the basics of neural networks and deep learning as well as the concepts and structure of the employed architectures were explained.

An algorithm for the measurement of droplets from segmentation mask was developed and its shortcomings and strengths as well as possible improvements were discussed.

The thesis also explored the use of transfer learning and self-supervised learning to improve the performance of the model in its application.
While some of the techniques weren't as successful as expected, the results of the experiments still showed promise and possible future improvements were suggested.

As it stands, the thesis achieved its goal of building a tool that could find use in current research and development projects, while also providing a basis for further investigation and refinement.

The use of SAW for the application of atomization is a promising technique that has the potential to be used in a variety of applications and the author hopes that the work presented in this thesis will contribute to the development of this technology.