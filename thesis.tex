\documentclass[
    headings=optiontohead,              % allows double headers
    12pt,                               % fontsize 
    DIV=13,                             % koma script diveider amount. tells koma how much of the site can be written to
    twoside=false,                      % if set to true, automatically formats as book style with different left and right pages
    open=right,                         % starting page on twosided texts 
    BCOR=00mm,                          % correction that accounts for the center of the pages being glued in
    toc=bibliographynumbered            % bibliography gets a number and is listed in the table of contents
]{scrreport}

\usepackage[english]{babel}                     % font that supports English
\usepackage{upgreek}                            % non-cursive Greek letters
\usepackage[stretch=10,shrink=10,protrusion=true,expansion=true,final]{microtype} % prettier block format
\usepackage{hyperref}                           % links for everything
\usepackage{color}                              % allows for setting in different colors
\usepackage[autooneside=false,automark]{scrlayer-scrpage} % page-style with "Kolumnentitel" (title of current chapter is displayed at the top)
\usepackage{amsfonts,amstext,amsmath,amsthm, amssymb} % better math mode (\mathrm and \text) and symbols
\usepackage[sb]{libertinus}                     % use the font libertinus (needs to be installed from the web)
%\usepackage[slantedGreek]{libertinust1math}     % math mode improvement for libertinus
\usepackage{siunitx}                            % physical units setting
\usepackage{icomma}                             % commas in lists get extra space if needed                        
\usepackage{xspace}                             % works to improve own commands and provides "\xspace"-command, that puts a space if needed
\usepackage{ifthen}                             % more control over non-obligatory parameters
\usepackage{titling}                            % get title values as macros
\usepackage[onehalfspacing]{setspace}           % control the spacing between lines and in enumeration lists
\usepackage[backend=biber, style=phys, biblabel=brackets]{biblatex} % citations with "modern" backend and an physics-accepted citation style
\usepackage{graphicx}                           % work with graphics 
\usepackage{ragged2e}                           % ragged-commands (when no block format is wanted)
\usepackage{pdfpages}                           % allows including of pdfs into this pdf
\usepackage{booktabs}                           % better table formatting
\usepackage{multicol}                           % allows for the definition of multi-columns in tables
\usepackage{multirow}                           % allows for the definition of multirow-tables instead of just multicolumn
\usepackage[section]{placeins}                  % provides the command "\FloatBarrier" to control the end of floatable regions for figures/tables
\usepackage{float}                              % provides the "H" option for forcing placement of a figure
\usepackage{floatpag}                           % make it possible for float-pages to not have a page number
\usepackage{url}                                % sometimes needed by biblatex, technically no longer needed
\usepackage{minted}                             % nice code highlighting (needs Python Package to compile!!)
\usepackage{mathtools}                          % more math control possibilities
\usepackage{eucal}                              % Different mathcal alphabet
\usepackage[autostyle=true]{csquotes}           % context-sensitive-quotes -> quotation marks that are set correctly for the context
\usepackage{physics}                            % bra-ket and more
\usepackage{nicematrix}                         % label row/cols on matrix
\usepackage{caption}                            % caption of different environments
\usepackage{subcaption}                         % subcaptions for figures
\usepackage{tikz}                               % Abbildungen zeichnen
    \usetikzlibrary{positioning}

\title{Measurement of Droplets in vapourised fluids using Machine Learning Techniques}
\author{Jan Claar}
\date{\today}

\input{texconf/config}                                  % another file that holds the package/document configuration
\input{texconf/format}                                  % another file that holds format information
\input{texconf/commands}                                % another file that holds predefined commands

\begin{document}
    \thispagestyle{empty} 
    \newcommand{\mail}{jan.claar@student.uni-augsburg.de}

\begin{titlepage} 
    \makebox[\textwidth][c]{\includegraphics[width=0.5\textwidth]{images/logo_uni_augsburg.jpg}} \color{dblue}
    \begin{center} \vspace*{2cm} \Huge \textbf{\thetitle}
        
        \vspace*{1.5cm} 
        
        \color{black} \textbf{Bachelor Thesis}

        \vspace*{1cm} 

        \normalsize submitted by\\ \LARGE \theauthor\\\vspace*{0.3cm} \normalsize on \thedate

        \vspace{1.8cm} 

        \color{black} \emph{Augsburg University}\\ \emph{Faculty ofApplied Computer Science}\\ \emph{Institute of Computer Science}\\ \emph{Chair for Machine Learning \& Computer Vision}
        \vfill

        \begin{tabular}{rl}
            1$^\text{st}$ Corrector: &Prof. Dr. Rainer Lienhart\\
            2$^\text{nd}$ Corrector: &Dr. Andreas Hörner\\ 
            Supervisor: & Daniel Kienzle\\ 
        \end{tabular} 
    \end{center}

\end{titlepage} 
    \tableofcontents
    \thispagestyle{empty} 
    \newpage 
    \setcounter{page}{1} 
    \pagestyle{scrheadings}

    \chapter{Introduction}
    \label{sec:introduction}
    The advancement of all major natural sciences has often gone hand in hand with advances in the respective measurement techniques used to formulate and validate hypotheses through experiments using the scientific method. 
Being able to measure key metrics in the system of interest is vital not only in scientific research, but also the developement of new technologies and products. 

The field of \emph{machine learning} and \emph{artificial intelligence} has seen an explosive growth over the past two decades and has become a major research focus for many in the discipline of data and computer science. 
With storage capacity and processing speed of computer hardware, such as specialized \textbf{G}raphics and \textbf{T}ensor \textbf{P}rocessing \textbf{U}nits (GPUs/ TPUs) experiencing similar advancements, the times are long gone when using AI to solve a problem was almost always unreasonable because of its large demand for data and computational resources. 

It is no wonder, then, that we ask how we can apply the power of machine learning to the field of measurement science. In fact, this question was asked even before the 2000s \cite{alippiArtificialIntelligenceInstruments1998}, at time when access to AI was much more restricted. Since then AI has been and will continue to be successfully used to improve measurement techniques by helping develop better sensors \cite{ballardMachineLearningComputationenabled2021}, processing raw sensor data or deriving meaning from already processed data. 

One way in which machine learning (in this case neural networks) can be used to improve the measuring process is by automating tasks that are difficult to solve through classical algorithms, but which are easy for a human to do. This is no surprise, since the idea of artificial neural networks in the first place is inspired by their biological counterpart. These problems often are very intuitively solved by the brain for a single instance, but require a large amount of effort when scaled. 
It is this aspect of machine learning that we hope to employ in our research. 

\textbf{S}urface \textbf{A}coustic \textbf{W}aves (SAW) can be produced by using electrodes to induce surface vibrations in piezoelectric substrates. These SAW chips have several useful applications, one of them being using them to vapourise fluids into very small droplets, which could be beneficial in fields such as e.g. medicine, for producing fine vapour of solutions of certain drugs to ensure optimal absorption in the body.
While researching and developing this technology it is important to obtain insight about the characteristics of the produced vapour, primarily droplet size and distribution. While there are already techniques to measure the size of droplets in a vapour, they all come with certain caveats. A simple solution is then, to take high speed image data of the vapour and measure the droplet size directly on the images, which is easy for one image, but constitutes an ardous task for each measurement run, if one desires to obtain any statistically robust data.

The goal of this thesis is, now, to explore the use of neural networks to automatically identify droplets in image data taken of SAW vapourised fluids and measure them with sufficient precision, as well as to compare this approach to using classical algorithms for image segmentation. Finally, a simple application will be developed which should find use in current research.

    \chapter{Theory}

    \label{sec:theory}
    \section{Physics}
    \label{sec:theory_physics}
    Although the thesis doesn't research the atomization technique itself in a large capacity, it is important to understand the physical concepts behind it, since they are responsible for the data that is used in this work.

With this in mind, the following sections will explain how Surface Acoustic Waves are produced and how they can be used to vaporize liquids as well as how the resulting droplets can be measured.
    \section{Surface Acoustic Waves and how they can be created}

True to their name, \emph{Surface Acoustic Waves (SAW)} are sonic waves that propagate along the surface of a solid material. 
The expression is an umbrella term for several kinds of waves that fall under this description, but the type of waves that is relevant to the application researched in this thesis is called \emph{Rayleigh Waves} \cite{mandalSurfaceAcousticWave2022}.

Rayleigh waves are a superposition of a longitudinal (P) and a shear vertical (SV) wave component and propagate through the surface of the substrate, with the amplitude of the particle motion decreasing exponentially with the depth of the substrate. 
Typically, the effective penetration in the direction normal to the surface is less than a wavelength. 
Because of how these longitudinal and vertical components, the waves produce an elliptical motion in the surface particles, with the plane of the ellipsis being parallel to the direction of propagation and normal to the material surface (see Figure \ref{fig:rayleigh}).

\begin{figure}[htbp]
    \centering
    \makebox[\textwidth][c]{\includegraphics[width=0.8\textwidth]{images/Screenshot-20230208172733-2080x997.png}}
    \caption{Depiction of Rayleigh-Waves propagating through a substrate. \cite{mandalSurfaceAcousticWave2022}}
    \label{fig:rayleigh}
\end{figure}

Other types of SAW waves include \emph{shear horizontal waves} and \emph{Lamb Waves}, however these wave types do not cause the strong vertical displacement which is of interest for the purpose of liquid atomization. From now on the term SAW will be used synonymously for Rayleigh waves.

For most applications, SAW are produced by fixing an \emph{interdigital transducer} (IDT) to a \emph{piezoelectric} substrate. An IDT is a type of electrode structure consisting of two sets of interleaved metal fingers, which are connected to the opposite poles of a voltage source. Applying an AC current to the IDT produces surface waves by exploiting the piezoelectric properties of the substrate.

A piezoelectric material generates an electrical voltage in response to applied mechanical stress, and conversely generates mechanical displacement in response to applied electrical voltage. 
The piezoelectric effect stems from the relative displacement of oppositely charged ions in a crystal with asymmetrical unit cells causing a displacement of the charge concentration and resulting in a larger electric dipole moment. 
If the dipoles in the crystal are aligned, the effect causes the whole crystal to be polarized, creating an electric voltage. 
Since the dipole moments in a crystal are typically only aligned locally in their respective Weiss domains, materials usually need to be poled in order to exhibit strong piezoelectric properties \cite{liLeadfreePiezoelectricMaterials2021}.
All piezoelectric materials also exhibit the reverse effect; applying a electrical field exerts electrostatic force on the dipoles which causes displacement of the ions.

\begin{figure}[htbp]
    \centering
    \makebox[\textwidth][c]{\includegraphics[width=0.4\textwidth]{images/Screenshot-20230208192523-1078x984.png}}
    \caption{The principle of SAW generation in a piezoelectric substrate. A voltage is applied to the two parts of an IDT, which causes mechanical tension in the substrate.}
    \label{fig:idt}
\end{figure}

Since the mechanical force caused by the phenomenon is parallel to the electrical field, applying AC voltage to the electrodes arranged in this interwoven pattern (see Figure \ref{fig:idt}) generates surface waves perpendicular to the direction of the electrodes.

The wavelength of the SAW is determined by the distance between the electrodes, which allows for a fine control of the wave frequency for each particular device.

Conversely, the same construction can be used to to pick up vibrations and convert them back to electrical voltage. 
This is the principle commonly used in SAW based sensors, which is only one of many useful applications of SAW technology like filtering in radio frequency technology or compact voltage transformers. 
In the next section the possible application for fluid atomization will be further discussed.
    \subsection{SAW as a method of fluid atomization}
\label{sec:saw_vapour}

Using SAW technology to produce small scale fluid atomization devices is not a recent idea. 
The concept was already demonstrated by \Citeauthor{kurosawaSurfaceAcousticWave1995} in 1995, but mass produced SAW-based atomizers are still not commercially available, likely due to challenges related to the precision engineering required for mass production.

However, because of the many possible applications of low-power, compact fluid atomizers such as inhalation therapy, thin film deposition or nanoparticle synthesis, there is still a substantial interest in improving the technology and optimizing it for production.

Depending on the boundary conditions various acoustofluidic effects can take place during the interaction between SAW and a fluid. \cite{winklerSAWbasedFluidAtomization2015a}
Depending on the geometrty of the liquid volume the physical phenomena at play can be very different and may result in different atomization regimes, some of which are not entirely understood yet. \cite{collinsAtomizationThinWater2012,huangExperimentalResearchSurface2022}
Describing these highly complex microfluidic phenomena in their entirety is beyond the scope of this work, which is why the following section will focus on the general principle of fluid atomization without going into detail about the governing equations.

When the SAW hits the liquid surface, it is diffracted into the liquid volume at the \emph{Rayleigh angle} $\theta_\text{R} = \sin^{-1}(c_\text{l}/c_\text{SAW})$, which depends on the speed of sound in the liquid $c_\text{l}$ and the propagation speed of the SAW $c_\text{SAW}$ and is $\sim\SI{22}{\degree}$ for water.
The acoustic radiation \emph{leaked} into the liquid causes a longitudinal pressure wave which leads to a bulk recirculation of the liquid known as \emph{acoustic streaming}.

This phenomenon is useful for a variety of applications, such as the mixing of liquids or the transport of particles in a liquid, but is not primarily responsible for the atomization process.
However it has an important influence on the geometry of the liquid volume, which has a significant impact on the droplet formation, as mentioned earlier.

The main driving force behind the atomization process is the capillary waves that form on the liquid surface.
The vertical surface displacement observed for Rayleigh waves is only around \SI{10}{\nano\meter}, however particles are accelerated at around \SI{e7}{\meter/\second\squared}.
If enough power is used, the capillary waves can be amplified to a point where they overcome the surface tension to break and form droplets.

\begin{figure}[htbp]
    \centering
    \makebox[\textwidth][c]{\includegraphics[width=0.7\textwidth]{images/Screenshot-20230211174610-984x494.png}}
    \caption{Schematic depiction of the SAW atomization process. The wave hits the and causes capillary waves on the liquid air interface. \emph{LN} stands for \emph{lithium niobate}, a commonly used piezoelectric material. \cite{aishaqiInvestigationSAWAtomization2009}}
    \label{fig:atomization}
\end{figure}

The exact relationships between the excitation frequency, the capillary wave frequency and the droplet size are not fully understood yet and different models haven been proposed over time.
The main influences in the relationship appear to be the liquid viscosity, the liquid surface tension and the liquid density. \cite{aishaqiInvestigationSAWAtomization2009,huangExperimentalResearchSurface2022}
The latest findings suggest that that the droplet size $D$ is in the same order of magnitude as the wavelength of the capillary waves $\lambda_\text{c}$, which seems to be inversely proportional to the excitation frequency $f$. \cite{collinsAtomizationThinWater2012}
$$
    D \sim \lambda_\text{c} \sim \frac{1}{f}
$$

Different SAW atomizers differ primarily in how they supply liquid to the atomization area, examples including a \emph{droplet-on-demand} system, where the liquid is supplied by a syringe pump, or a \emph{continuous flow} system, where wetted laboratory paper is placed  on the substrate and the liquid is supplied by a reservoir, with atomization taking place at the miniscus that forms at the edge of the paper. \cite{winklerSAWbasedFluidAtomization2015a}

The method of liquid supply researched at our institute uses a small capillary channel that is mill-cut directly into the substrate, which has direct contact to a liquid reservoir, filling itself using the capillary effect.
This method is very simple and does not require any additional components, which makes it very suitable for mass production.
Ongoing research explores the influence of channel geometry on the atomization process. \cite{kapplAkustischInduzierteVernebelung2022}


    \section{Droplet Measurement Techniques}
\label{sec:measurement_techniques}

For most applications of microparticle aerosols droplet size is a key variable that needs to be controlled precisely to achieve the best results.

For example, in the case of a drug delivery system, the size of the droplet determines the amount of drug that is absorbed in the lungs, as well as the locations it is deposited in, with smaller droplets reaching more deeply into the lungs, but releasing a smaller amount of drug per unit surface, while larger droplets are more likely to be deposited in the upper airways, but release a larger amount of drug per unit surface area \cite{labirisPulmonaryDrugDelivery2003}.

Therefore, when developing such a system accurate measurement of the droplet size is essential. There are a number of techniques that can be used to measure the size of a droplet, but most of them have some drawbacks or are not suitable for use in this particular application. 
This section gives an overview on the most common techniques as well as our method.

\paragraph{Laser diffraction techniques} make use of the fact that the diffraction angle of light passing a sphere is inversely proportional to the diameter of the sphere, with large particles scattering the light at smaller angles and smaller particles scattering the light at larger angles, according to \emph{Mie theory} for electromagnetic wave scattering \cite{drakeMieScattering1985,wriedtMieTheoryReview2012}.

Devices pass a laser beam through the vapour and record the angular scattering intensity, which is then analyzed to calculate the size of the particles. 

Theoretically, laser diffraction techniques are very accurate and able to measure particles as small as \SI{0.1}{\micro\meter} and have been used to measure SAW atomized particle distributions previously \cite{aishaqiInvestigationSAWAtomization2009}. 
However, a company which is in a cooperative relationship with our institute for the purpose of researching SAW atomization has reported inconsistencies in their use of a commercially available device, which is why we have chosen to investigate other techniques.
Another reason for choosing other techniques is the potentially prohibitive pricing of the laser diffraction devices in this stage of the research.

\paragraph{Phase Doppler Particle Analysis (PDPA)} uses two laser beams that cross each other in a volume with an ellipsoidal cross section in which an interference pattern between the lasers is produced, forming fringes of differing light intensity \cite{hollPARTICLEDEPOSITIONVELOCITIES}.
When a particle passes through the volume, this fringe pattern is scattered, which produces pulses of light called \emph{doppler bursts} that can be picked up by two or more well placed photo detectors.
Information about the particle can be extracted from the phase difference of the doppler bursts picked up by the different detectors.

PDPA is a very accurate technique, but requires very large and specialized equipment, with commercial devices facing similar drawbacks to laser diffraction devices in terms of price and size.

It is also known to produce inaccurate readings for inhomogeneous particles or other disturbances in the measurement environment, which requires a more complicated set up \cite{sijsDropSizeMeasurement2021}.

\paragraph{Image Analysis} 
The technique used during our research up until this point employs a more pragmatic approach. A high speed camera in combination with a microscope objective lens and a parallel light source is used to record images of the aerosolized particles as they are being produced. The images were then analyzed manually one by one to determine the size of the particles. While being very flexible and easy to calibrate, this is an extremely tedious and time consuming process.

This is the aspect of the research that this thesis aims to improve upon by utilizing machine learning techniques to automate the process of droplet identification and subsequent size measurement.

Using this kind of image analysis method is not unprecedented, as \Citeauthor{sijsDropSizeMeasurement2021}\cite{sijsDropSizeMeasurement2021} have demonstrated a similar approach in 2021, where they use image processing techniques to identify and measure droplets in image data. 
However, they do not disclose the details of their image processing pipeline, making it unclear if they employ machine learning models for this purpose.
While the idea behind their approach seems to be similar, \Citeauthor{sijsDropSizeMeasurement2021} report a minimum droplet size of \SI{150}{\micro\meter} for their technique, which is one to two orders of magnitudes larger than our expected droplet size of \SIrange{1}{30}{\micro\meter}.
As a result, their images seem to have much less noise and uneven lighting conditions, which makes droplet identification easier.
Our method should therefore be more applicable to the kind of images that we are working with, but could be used for the same purpose when measuring larger droplets.



    \newpage
    \section{Computer Science}
    \label{sec:theory_compsci}
    As stated in the Introduction, the goal of this thesis is to use a neural network to identify the droplets in the captured image data to measure their size. 
This problem lies in the domain of \emph{instance level image segmentation}.
However, since individual droplets almost never overlap and droplets themselves have a relatively uniform shape, as long as we know what kind of object each pixel belongs to we can identify the different droplet instances after inference.
This allows us to simplify the problem to \emph{semantic segmentation}, which is a much easier task.
The employed network is a \emph{(Deep-) Convolutional Neural Network} (DCNN). 

In the following section, a brief overview of the basics of neural networks, how they work in general and how they can be adapted to a certain task is given. It also explains some classical image processing techniques that might be used on a problem like this one in the absence of neural networks.

\section{Basics and building blocks of neural networks}
\label{sec:building_blocks}

The idea of artificial neural networks came long before the capability to employ them as a tool and stems from research in trying to model a biological neuron's function. Biological nerve cells receive signals from several other neurons and then send their own signal based on the strength of their inputs. Translating this behaviour to a mathematical model brings us to the \emph{single layer perceptron}.

\paragraph*{Perceptron}

\begin{figure}[htbp]
    \makebox[\textwidth][c]{
    \tikzset{basic/.style={draw,fill=blue!20,text width=1em,text badly centered}} 
    \tikzset{input/.style={basic,circle}} 
    \tikzset{weights/.style={basic,rectangle}}
    \tikzset{functions/.style={basic,circle,fill=blue!10}}

    \begin{tikzpicture} 
        \node[functions] (center) {}; 
        \node[below of=center,font=\scriptsize] {Activation function}; 
        \draw[thick] (0.5em, 0.5em) -- (0,0.5em) -- (0,-0.5em) -- (-0.5em,-0.5em); 
        \draw (0em,0.75em) -- (0em,-0.75em); 
        \draw (0.75em,0em) -- (-0.75em,0em); 
        \node[right of=center](right) {}; 
            \path[draw,->] (center) -- (right); 
        \node[functions,left=3em of center] (left) {$\sum$}; 
            \path[draw,->] (left) -- (center); 
        \node[weights, left=3em of left] (2) {$w_2$} -- (2) node[input,left of=2] (l2) {$x_2$};
            \path[draw,->] (l2) -- (2); 
            \path[draw,->] (2) -- (left); 
        \node[below of=2] (dots) {$\vdots$} -- (dots) node[left of=dots] (ldots) {$\vdots$};
        \node[weights,below of=dots] (n) {$w_n$} -- (n) node[input,left of=n] (ln)
    {$x_n$}; \path[draw,->] (ln) -- (n); \path[draw,->] (n) -- (left);
        \node[weights,above of=2] (1) {$w_1$} -- (1) node[input,left of=1] (l1)
    {$x_1$}; \path[draw,->] (l1) -- (1); \path[draw,->] (1) -- (left);
        \node[weights,above of=1] (0) {$w_0$} -- (0) node[input,left of=0] (l0) {$1$}; 
            \path[draw,->] (l0) -- (0); 
            \path[draw,->] (0) -- (left); 
        \node[below of=ln,font=\scriptsize] {inputs}; 
        \node[below of=n,font=\scriptsize]{weights}; 
    \end{tikzpicture}
    }
    \caption{Displayed is a schematic of the single layer perceptron. $x_i$ are the inputs that get multiplied by the weights $w_i$. $w_0$ is also called the \emph{bias}. The product of weights and inputs is then summed up and passed to an activation function that computes the output of the perceptron. \cite{m0nhawkAnswerTikZDiagram2013}}
    \label{fig:perceptron}
\end{figure}

The perceptron displayed in \ref*{fig:perceptron} computes its output like
$$
    o = f\left(w_o + \sum_{i} w_i x_i\right)
$$
where $f$ is the activation function, which is some differentiable nonlinear function. State of the art networks use mostly the \emph{Rectified Linear Unit} (ReLU) \cite{nairRectifiedLinearUnits2010} as their activation function, which has proven to be preferable to its predecessor, the sigmoid function, and is computed as such:
$$
    \text{ReLU}(x) = \begin{cases}
        x, &x>0\\
        0, &\text{else}
    \end{cases}
$$

Naturally, modern networks contain much more than one neuron and instead contain many such nodes, that are interconnected with each other. 
These neurons are typically organized in \emph{layers} and form, in the case of feedforward networks, an acyclic directed graph, where neurons of one layer are only connected to the neurons in layers further down the pipeline. Often the layers between output layer and input layer are called \emph{hidden layers}, since the intermediate representation of the input data they compute is not immediately of interest.

While a single neuron doesn't have the capability to solve complex problems, it has been show that a network with as little as one fully connected hidden layer (all neurons of the previous layer output to all neurons of the following layer) can function as a \emph{universal approximator} for functions between two Euclidean spaces\cite{hornikMultilayerFeedforwardNetworks1989} or indeed any $L^p$-Space\cite{parkMinimumWidthUniversal2020}. This means if there is a mathematical correlation between our desired input and output, so long as both can be represented in these spaces, we can find a neural network that approximates this correlation very well.

While this is an extraordinary finding, modern neural networks often place much more emphasis on the number of layers (depth) than the number of neurons in one layer (width) of a network, hence the terms \emph{deep learning} or \emph{deep neural networks}, since deep networks are easier to train than wide networks. However, the size of a network's parameters and with that its computational complexity grows fast when using many fully connected layers. One tool that enables the use of very deep networks is the \emph{convolutional layer}.

\paragraph*{Convolutions and convolutional neural networks}

Similarly to a discrete 2d convolution in mathematics, a convolutional layer in a neural networks moves over the input space with a comparatively (to input dimensions) smaller kernel and computes the output by multiplying the weights stored in the kernel with the corresponding input values.

\begin{figure}[htbp]
    \makebox[\textwidth][c]{
        \includegraphics[width=\textwidth]{images/att_00028.png}
    }
    \caption{Depiction of how a convolution computes its outputs. A $4\times 4$ input convolved with a $3\times 3$ kernel produces a $2\times 2$ output. Since no padding is used this constitutes a valid convolution. \cite{Att00028Png}}
    \label{fig:convolution}
\end{figure}

The example seen in Figure \ref{fig:convolution} shows only one channel and a procedure that employs no padding, leaving the output smaller than the input (called a valid convolution). Other than valid convolutions, padding the input by $\lfloor\frac{k}{2}\rfloor$ or $k - 1$, where $k$ is the size of the kernel, leaves us with a \emph{same} (same size as input) or \emph{full} (larger than input) convolution. Additionally, in reality inputs to a layer often have many channels and the layer should be able to produce an arbitrary amount of output channels. In this case there exists a kernel $K_{\text{in}, \text{out}}$ ($K\in \mathbb{R}^{C_\text{in}\times C_\text{out}\times W_\text{K} \times H_\text{K}}$) for each combination of input channels $C_\text{in}$ and output channels $C_{\text{out}}$ whose convolution results are then added up for each output channel, so that for an input $I\in \mathbb{R}^{C_\text{in}\times W \times H}$ the output $O\in \mathbb{R}^{C_\text{out}\times H' \times W'}$ is calculated like
$$
    O_j = \sum_{k=1}^{C_\text{in}} K_{k,j} * I_k\quad,
$$
where $*$ is the convolution operation. There are other approaches to handle multi channel scenarios, but this is the simplest one and is employed in the thesis.
Another parameter which is important in the context of convolutions is the \emph{stride}, which determines how many units the kernel is moved over the input in each step. The example shown in Figure~\ref{fig:convolution} has a stride of 1, but often, a stride of 2 or more may be used when utilizing convolutions to downscale inputs.

Using a convolution layer has several advantages over using fully connected layers. Using a convolution is a reduction in \emph{learnable parameters} (see section \ref{sec:training}) over a fully connected layer for identical input and output dimensions. This reduces computational resources required and thereby speeds up training. It introduces the inductive bias that positionally close data points are correlated and relevant to each, which is reasonable especially in tasks concerning computer vision. 

For example, take a model that tries to detect circles in an image. It makes sense to assume that features which make up a circle are positionally close to each other. Convolutions are also translationally invariant, because they use the same kernel at each position of the images, regardless of the position. 
In this example that means that for two identical circles at different locations in the input, the identical output is produced at their corresponding position, which is not the case for fully connected layers, where very different weights might me learned for different positions in the picture if data is insufficiently random.

Many architectures in the computer vision field now purely employ convolutional layers (\emph{fully convolutional neural networks}) or use very few fully connected layers only to compute the final output of the network at the end. 

There are additional constructs that can be used in neural networks, such as \emph{pooling} layers, which also move a kernel over the input like a convolution, but instead take the average or maximum of the covered elements as their output. As such, the pooling layer contains no learnable parameters. They are generally used to reduce the dimensions of the feature map to further speed up training time and make the model more stable by promoting invariance to small perturbations in the input and increasing the receptive field of a neuron.

    \subsection{Semantic Pixel Level Image Segmentation Task}
\label{sec:imseg}

With the knowledge of how neural networks perform their computation in general it is also necessary to think about how a specific task can be accomplished by crafting a specialized representation of the problem, which networks can be applied to.
There are several categories of typical problems within computer vision, that have different kinds of encodings for their solution space. The task which this thesis is trying to solve lies in the category of \emph{Semantic Pixel Level Image Segmentation}, where the network input is an image and the network is supposed to assign each pixel one of a predetermined set of \emph{classes} to which the object the pixel is part of belongs to. For example in the case of the \emph{Cityscapes Dataset}\cite{cordtsCityscapesDatasetSemantic2016a}, the data consists of images of inner city traffic situations from the perspective of a car, and the classes are subjects typically found in such scenarios like e.g. \emph{car}, \emph{road}, \emph{person}, \emph{building} etc.

\begin{figure}[htbp]
    \centering
    \begin{tabular}{ll}
        \includegraphics[width=0.45\textwidth]{images/aachen_000029_000019_leftImg8bit.png}
        &
        \includegraphics[width=0.45\textwidth]{images/aachen_000029_000019_gtFine_color.png}
    \end{tabular}
    \caption{A sample from the dataset \emph{Cityscapes}. On the left is the image that acts as an input for the network and on the right is the corresponding ground truth label mask showing different classes in different colors.}
    \label{fig:cityscapes_smpl}
\end{figure}

A model for this task would now output a map with the same spatial dimensions (e.g. $M\times N$) as the image and a number of channels $C$ equivalent to the number of possible classes. Each output pixel consists of a vector $\mathbf{x}\in \mathbb{R}^C$ where each entry $x_i$ corresponds to how much the network thinks the pixel belongs to class $i$. The channel with the highest value is the class the network has classified the pixel as.

These outputs can now be normalized to probabilities by applying the \emph{softmax} function to them or converted to a segmentation mask by using the \emph{argmax} function.

To evaluate how well a network performs on the segmentation task, the \emph{Intersection over Union} (IoU) for each class is computed, which if $A$ is the set of pixels labeled as a certain class by the network and $B$ the set of pixels which should actually belong to that class is calculated as follows:
$$
    \text{IoU} = \frac{\left|A\cap B \,\right|}{\left|A\cup B \,\right|}
$$
This metric is very informative to look at per class, but typically the mean over all classes (mIoU) is used to evaluate the model as a whole.\\

A big problem in developing AI solutions to this kind of task is its huge demand for varied data to train models on. High quality annotated data is very time and cost consuming to produce, for example according to the Cityscapes team, one annotated image such as seen in Figure \ref{fig:cityscapes_smpl} took \SI{90}{min} to create. 
Further in this thesis techniques to mitigate this problem and achieve good results on little annotated data will be discussed.

    \subsection{Training neural networks}
\label{sec:training}



    \chapter{Architectures}
    \label{sec:architectures}
    When approaching a problem with a machine learning solution in mind, the first decision one has to deliberate on is the choice of network architecture used. 

In the case of neural networks, the word \emph{architecture} means which kind of layers are connected to each other in what order to produce the output of the network.

The landscape of machine learning architectures has become incredibly diverse, with improvements being made constantly.
Since some architectures are better suited for certain problems or priorities, which architecture one chooses has a significant impact on the results.

In this section, an overview over the architectures used in the thesis and their main ideas will be given.
    \section{The U-Net}

The \emph{U-Net} is a network achitecture first proposed by \Citeauthor{ronnebergerUNetConvolutionalNetworks2015} in 2015 for application in biomedical segmentation tasks, for example segmenting cell borders in microscopic images of HeLa cells. 
The challenges the authors were facing at the time are similar to the ones in this thesis, where training data for biomedical segmentation tasks was scarce, which is why the U-Net was makes sense as a starting point for the problem of droplet segmentation.

It is a fully convolutional network, which were on the rise to surpass older state of the art neural networks for classification tasks at the time. 
The network employs an \emph{encoder-decoder} structure, meaning the architecture consists of a contracting path and a expanding path, the \emph{encoder} and \emph{decoder} respectively, as seen in Figure \ref{fig:unet}.

\begin{figure}[htbp]
    \includegraphics[width=\textwidth]{images/unet.png}
    \caption{The original U-Net architecture proposed in \cite{ronnebergerUNetConvolutionalNetworks2015}, for an example of a $572\times 572$ input, with the number of channels of the layer written above the blue boxes representing the feature map after each layer passthrough. The legend shows which operation was used between each feature map.}
    \label{fig:unet}
\end{figure}

The encoder path computes a number of features at four different spatial input sizes with two $3\times 3$ convolutions followed by a ReLu, which are then downsampled by a $2\times 2$ max-pooling layer. 
With each of these blocks, the spatial resolution is halfed along each axis, while the number of feature channels is doubled.

The features output by each block depend on different sized regions of the initial input. 

The values of blocks with high spatial resolution consider only small patches in the input image, while a lot of pixels influence each value for the lower blocks. 
The large \emph{perceptive field} of the lower blocks allow for rich contextual information to be encoded in their feature maps.

In a classification problem, such contracing networks would be used by feeding the highly dense information of the last encoder block to a few fully connected layers which make the final classification decision. 
In this case, since a classification for each pixel is needed, the encoded information must now be scaled back up to the desired resolution.

The decoder path is symmetrical to the encoder path, halving the number of feature channels in each block and upsampling the spatial resolution by using $2\times 2$ \emph{transposed convolutions}, which act like a backwards pass through a normal convolution.
After the last upsampling block a final $1\times 1$ convolution is used to decide the final class for each pixel.

However, as is, this structure has a key flaw when it comes to creating segmentation masks. While it is feasable to deduce general locations of objects from the highly contextual encoder features, it is difficult to infer their exact boundaries, because the spatial resolution has been compressed so much. 

This is where another key aspect of the U-Net architecture comes in. 
By concatenating the outputs of each encoder block to the input of their respective decoder block, we allow the decoder to not only utlize the context information provided by the last enocoder layers, but also the very localized features of the high resolution blocks.

This combination allows the U-Net to predict the correct classes along with their precise spatial location.

The U-Net outperformed its predecessors by a large margin and has since been developed further. Its concepts still serve as the basis for popular segmentation networks.\\

One way in which the original U-Net architecture is often modified, is using more sophisticated models for the encoder module, which is the approach taken in this thesis. 
Obtaining better features has proven itself to lead to better overall results, so investing more capability in the encoder structure is often a good idea. 

This also comes with the added benefit that pretrained weights for common feature extractors like the \emph{ResNet} (more in \ref{sec:resnet}) are readily available. 
    \section{The ResNet}
\label{sec:resnet}

The \emph{Residual Network (ResNet)} is a deep neural network architecture first introduced by \Citeauthor{heDeepResidualLearning2015} in 2015, which addresses the seemingly paradoxical circumstance that adding more layers to a deep neural network would degrade its performance compared to a shallower network, even though the solution space of the shallower model is a subspace of its deeper counterpart. 

Working from the idea that adding more layers to a network should not produce a higher error, since the additional layers could potentially just learn identity mappings, the authors surmised that the problem lied with the difficulty for the optimization process to learn the underlying desired mapping for a set of layers if the ideal mapping is closer to an identity mapping than a zero mapping.

The solution to this, proposed by ResNet, is to utilize residual connections between the input and output of a layer group, by directly adding the input to the output. 
Instead of having to learn the desired output mapping $\mathcal{H}(x)$, the layers now have to fit the residual mapping $\mathcal{F}(x) := \mathcal{H}(x) - x$, which the authors argue is easier for the optimizer to do.

\begin{figure}[htbp]
    \makebox[\textwidth][c]{
        \includegraphics[width=0.4\textwidth]{images/Screenshot-20230207134546-1894x1101.png}
    }
    \caption{A depiction of a \emph{residual block} employed in the ResNet network architecture. The input of a group consisting of several layers is added directly to the output. The weighted layers can be linear layers but in practice, mostly convolutional layers are used. The blocks also consist of 3 layers the majority of the time. Image taken from \cite{heDeepResidualLearning2015}}
    \label{fig:resblock}
\end{figure}

Employing these \emph{residual blocks}, the convergence rate of the network is significantly improved, without adding any computational complexity. 
This enables the use of very deep networks, gaining substantial accuracy from the added layers.

Apart from the residual connections, the ResNet still follows a typical encoder architecture, with each block reducing spatial resolution, while increasing the number of feature channels.
There are several versions of the ResNet architecture, differing mainly by the number of layers used. 
\Citeauthor{heDeepResidualLearning2015} looks at networks with up to 152 layers (ResNet152), but deeper networks have been explored. 

Since the residual connection allow the network to propagate information from the shallower layers to the deeper layers more easily, it may also combat the problem of vanishing/exploding gradients, which were a challenge for increasingly deep models at the time, however the authors argue that this was already sufficiently addressed by regularization techniques such a \emph{Batch Normalization}. \cite{ioffeBatchNormalizationAccelerating2015}

ResNet was able to significantly outperform its state-the-art predecessors at the task of image classification and is widely used today, often as part of a larger model ensemble.

In this thesis, \emph{ResNet34 D} is used in most experiments, which makes several minor improvements over the regular ResNet.
This includes employing an average pooling layer before the strided $1\times 1$ convolution present in the identity path of each downsampling block to include all datapoints, switching the stride of convolutions in the residual path for the same reason and replacing the $7\times 7$ initial downsampling convolution with three $3\times 3$ convolutions.
These modifications see an improvement in classification error while only increasing computational cost slightly. \cite{heBagTricksImage2018} 

\begin{figure}[htbp]
    \makebox[\textwidth][c]{
        \includegraphics[width=0.9\textwidth]{images/Screenshot-20230207145624-1651x952.png}
    }
    \caption{Illustration of the three improvements made by ResNet34 D, with the blue boxes representing the changes made. Left depicts the change made to the residual path of the downsampling block, middle illustrates the changes made to the initial block of the network and right explains the changes to the identity path. $s$ represents the stride of the operations, with it being one if ommited. Image taken from \cite{heBagTricksImage2018}}
    \label{fig:resnet34d}
\end{figure}

\begin{figure}[htbp]
    \makebox[\textwidth][c]{
        \includegraphics[width=0.2\textwidth]{images/Screenshot-20230207144959-265x1725.png}
    }
    \caption{Diagram for the ResNet34 model. Curved arrows represent a residual connection, with dotted arrows meaning the connection also downsamples the input. The four encoder blocks are colored differently. For use as an encoder, the fully connected layer at the end is omitted. Image taken from \cite{heDeepResidualLearning2015}}
    \label{fig:resnet34}
\end{figure}

    \chapter{Techniques used to improve model accuracy}
    \label{sec:techniques}
    Basic steps such as optimizing parameters for training are not the only methods available to try to improve the performance of the resulting model. 
Other techniques may be used to achieve better results, either by facilitating learning in some way or exploiting more data. 
In the following sections, the techniques employed during model development in this thesis will be explained more in depth.
    \subsection{Transfer Learning}
\label{sec:transfer_learning}

Coming back to the human learning analogy, it is often the case that knowledge or skills gained in one field can be applied to other fields, without having a lot of experience in the new fields. For example one might take their knowledge about composition from drawing to apply it to taking good pictures in photography. Although the technical details of both fields are different, they have some shared concepts, which help a person proficient in one to perform well in the other. That same person might also be able to improve in this new discipline faster than others, who do not have any applicable prior knowledge, because they can focus on learning other key skills needed to excel.

\emph{Transfer leaning} is the idea to apply this concept to neural network training. From a high level perspective, what happens during computation in a neural network, is that each layer transforms the input into an \emph{intermediate representation} of the data, which extracts different features present in the input. For example, one layer might learn to detect edges, which the next layer then combines to knowledge about the location of corners. As you can imagine, these types of features are useful for recognizing objects, such as a tennis ball, in a picture. It also stands to reason that these same features would be useful to detect circular droplets. The extracted features are rarely this clear cut or human-understandable as in this example, but the point still stands. 

Using a model trained on one dataset and using these weights as a starting point to fine tune the model on the actual problem dataset is also called \emph{pre-training}. This is especially useful if the data for the target problem is scarce, as pretraining is often done on very large datasets such as \emph{ImageNet}, which features over 14 million images (smaller subsets available) annotated for \emph{image classification}. Pretrained weigths for popular architectures are readily available for download and can be used as is if the output format of the network fits the new data, by switching out the final layers and learning only the classification or by using them as part of a bigger network, for example using them as a feature extractor in an \emph{encoder-decoder} architecture (see \ref{sec:architectures})

There are other forms of transfer learning, such as \emph{distillation}, where the concept is to use a complex model that is well adapted to the task to train a comparatively smaller model. The larger model has a higher knowledge capacity, but not all of this capacity has learned important knowledge. When training the smaller model on the soft outputs (before argmax) of the larger model, it may learn correlations which it might not have been able to learn on its own given its limited capacity. The smaller network could then be deployed instead of the larger one to save computation resources on weaker hardware.

In this thesis pre-training is used in two places. Firstly, for the encoder module of the U-Net architecture, a ResNet that is pretrained on the ImageNet dataset is used. Secondly, the experiments examine if pretraining the model on the Cityscapes dataset helps to improve performance on the vapour image dataset.
    \section{The mean teacher approach to semi supervised learning}
\label{sec:mean_teacher}

As mentioned already, the starting hurdle for building machine learning solutions is obtaining enough high quality annotated data. 
For image segmentation tasks, this means annotating segmentation masks, which are notoriously time consuming to produce. 
In our case, no annotated data was present in the beginning and it would be unreasonable to spend a large amount of time to annotate a lot of images, so only a limited amount of labeled samples were produced. 
In contrast, unlabeled images can be obtained very quickly and in large amounts, as one good measurement run can produce hundreds to thousands of images (even though not all of them may be usable).

Thus it would be very beneficial if we could make use of this unlabeled data in some way during training. 
Approaches to learning that utilize both labeled and unlabeled data are called \emph{semi-supervised learning}, in contrast to \emph{(fully-)supervised} or \emph{unsupervised learning}, where only labeled or no labeled data is used respectively. 
One example for such a procedure is called the \emph{Mean Teacher approach} \cite{tarvainenMeanTeachersAre2018}.

The main idea of the Mean Teacher approach it to use two models, one \emph{student} and one \emph{teacher}, and have the student learn from examples produced by the teacher.

The reason unlabeled data is not directly usable for model training is that no classification cost can be applied to the outputs, as the target is undefined.
While there is no way to automatically generate a 100\% accurate target, as this is essentially what we try to train our model to do, when we look at it the other way around, we can use a model to approximate the target to a degree.
However, using the same model we want to train to simply approximate its own training samples would not provide any benefit.
Two steps are are used in the Mean Teacher method to still make use of these kinds of \emph{pseudolabels}.

As mentioned in \ref{sec:training}, augmentation and regularization techniques have the purpose of enabling the model to learn the concept of its target function more broadly, since small variances in the input should still produce a similar output. 
This can be applied not only when comparing the output of the model to the ground truth, but also when comparing the outputs of the model for the same input data, but different levels of noise. 
If the model has properly learned the correct abstractions, its prediction should be similar for slightly different inputs, even if the outputs are less than perfect.
What this means concretely, is that the method uses the teacher to make a prediction on an input without any added noise and then computes student predictions on the same input to which augmentations have been applied.
In general, the teacher should have an easier time to predict accurate labels for a sample without noise, so a slightly better approximation of the theoretical correct labels should be produced. 
A \emph{consistency cost} can then be applied between student and teacher predictions which are then used in the same way as the \emph{classification cost} to update student weights.
Note that if the augmentations transform the input's geometry the teacher predictions have to be transformed accordingly.

Another way to improve the relative quality of the pseudolabels is to improve the teacher model they are generated by.
This is where the second key concept of the method comes in.
Instead of using the same weights for both student and teacher models, the \emph{exponential moving average} (EMA) of the student weights are used for the teacher.
What this means is, if $\theta_t$ are the weights of the student at step $t$ and $\theta'_t$ are teacher weights at the same point, the teacher weights $\theta'_{t+1}$ are updated as follows:
$$
    \theta'_{t + 1} = \alpha\theta'_t + (1 - \alpha) \theta_t
$$
where $\alpha$ is a smoothing coefficient called the \emph{EMA decay}.

The EMA of a model has been observed to be slightly better than the model itself on average, as well as being more stable during the training process, since changes to the student model are only adapted slowly. Combining this with the augmented student inputs makes it possible to extract a lot of improvement from unlabeled examples, but can also be applied during training steps with labeled data. 

\begin{figure}[htbp]
   \makebox[\textwidth][c]{
        \includegraphics[width=\textwidth]{images/mean_teacher.png}
   }
   \caption{Schematic depiction of the Mean Teacher method. The graphic illustrates a single training step with one labeled sample for the image classification task. The classification loss is applied between the ground truth and the model predictions, while the consistency cost is computed between the soft outputs of both models. After updating the student weights with backpropagation, new teacher weights are computed as the EMA of the student weights. Steps with unlabeled samples simply omit the classification cost. Image taken from \cite{tarvainenMeanTeachersAre2018}}
   \label{fig:mean_teacher}
\end{figure}

During training, a relative weight $\lambda$ is applied between classification loss $L_\text{class}$ and consistency loss $L_\text{cons}$, so that the overall loss is given by:
$$
    L = L_\text{class} + \lambda L_\text{cons}
$$
The weight of the consistency loss should start low and be slowly ramped up over the period of the training. 
This is necessary since initially, the teacher model may produce very bad outputs.
These outputs might also differ a lot from the student outputs, since no concepts haven been learned yet. 
If the consistency is given too much weight the model will prioritize being consistent with the teacher over predicting the correct classes for the labeled examples, potentially hurting or preventing any training progress. 

Another key aspect of the method is considering which augmentations are chosen for the student inputs \cite{tarvainenMeanTeachersAre2018}. To produce a large enough difference between outputs, strong augmentations are necessary to achieve the best improvements. The augmentations should also be well tailored to the dataset that is being trained on. 

Lastly, as mentioned above, the method introduces the hyperparameter $\alpha$ into the training, which has a big influence over the improvement that can be achieved. If it is too low, the benefits of using the EMA may not be present as much, since the teacher is too similar to the student. If it is too high, the teacher may not be updated quickly enough to incorporate the students improvement. 
Generally a value of $\alpha$ between \numrange{0.99}{0.999} is used, which has been shown to work well in practice.

The applicability of the mean teacher method will be more closely examined during the experiments of the thesis, with the goal to improve overall accuracy as well as generalization of the model produced. 

    \chapter{Droplet detection}
    \label{sec:detection}
    Before discussing the experiments conducted in this work, it is important to understand in which context the final model will be used to process experimental data. As well as the limitations of the model and the experimental setup.

\section{Detection and measurement algorithm}
\label{sec:algorithm}

The measurement process consists of several steps, each employing different measures to improve meaurement accuracy and reduce computation time.

\paragraph{Step 1: Image normalization} is done on the raw image data. This step is done to improve visibility of the droplets in the image and make a first guess as to which images actually contain droplets. 
The normalization is done batched by calculating the mean image $\mathbf{m}\in\mathbb{R}^{H\times W}$ as well as the mean greyscale value $\mu\in\mathbb{R}$ of the batch $\mathbf{B}\in\mathbb{R}^{N\times H\times W}$, substracting $\mathbf{m}$ from each image in the batch, adding $\mu$ to each pixel and then mapping the resulting values to the interval $[0,255]$.
This process must be done batched because of memory constraints, but batching the images also has the advantage that images in one batch are more likely to be similar to each other in terms of lighting conditions, which makes the output more consistent.

The mean image subtraction is done to remove camera and lens artifacts such as hot pixels or dust from the images.
This is not primarily done to improve the models detection accuracy, but in order to filter out images that are not suitable for further processing.
The metric for deciding if an image contains any strucure that could be a droplet is the \emph{Michelson contrast}, which is defined as 
$$
    C = \frac{I_\text{max}-I_\text{min}}{I_\text{max}+I_\text{min}},
$$
where $I_\text{max}$ and $I_\text{min}$ are the maximum and minimum luminance values in the image.
Images aren't considered for further processing if $C$ is below a certain threshold.
The reason artifacts need to be removed is that the they are very dark compared to the background, which means even images without any droplet structures will have a high Michelson contrast.

\paragraph{Step 2: Droplet detection} is done by a model trained to identify droplets with dark borders and bright centers as in focus and segment the images into three classes, \emph{droplet border}, \emph{droplet inside} and \emph{background}.
Segmenting the images into three classes instead of two is important in the next step, since it helps to detect errors the model made in the segmentation process.

Details on model training and evaluation can be found in \ref{sec:experiments}.

\paragraph{Step 3: Droplet measurement} is done by first using the \mintinline{python}{label()} from the \mintinline{python}{skimage.measure} library to differentiate between every connected area labeled as \emph{droplet border} in the segmented image. 
As a first filtering step, all areas that are smaller than a certain threshold are discarded, since no droplets below a certain size can be expected to fit the criteria for being in focus.
For each remaining area, the locations of all pixels is averaged to locate the center of the droplet.
Since the droplet is approximately circular, the average distance of all pixels from the center is calculated and used as a measure for the radius of the droplet.
Because the border of the droplet has a certain width and sometimes extends beyond the actual droplet, only the values between the 85th and 95th percentile of the distances are used to calculate the radius.

\section{Common problems and limitations}
\label{sec:limitations}

The model used to segment the image is not perfect, which means that sometimes areas which do not belong to a droplet are labeled as \emph{droplet border} or \emph{droplet inside} or the model fails to detect a droplet that is actually present in the image.
In this section, the most common errors are discussed and possible solutions are proposed, some of which are implemented already.

\paragraph{The model labels an out of focus droplet or other noise partially with border and center labels.}
\label{sec:partially_wrong}
This is the most common error observed when applying the model to experimental data examples for which can be seen in \ref{fig:partially_wrong}.
The model will assign a border label to the dark pixels of an out of focus droplet or a center label to brighter areas of the image that are not part of a droplet, possibly even both for the same structure.
However, this will very rarely happen in a way in which the border region completely surrounds the center region in a closed shape, which can be used to filter out these errors.

\begin{figure}[htbp]
    \centering
    \begin{tabular}{ll}
        \includegraphics[width=0.4\textwidth]{images/bad1.png}
        &
        \includegraphics[width=0.35\textwidth]{images/bad2.png}
    \end{tabular}
    \caption{Example images showing a labeling error as described in \ref{sec:partially_wrong}. Blue pixels represent \emph{droplet border} labels, and pink pixels represent \emph{droplet inside} labels.}
    \label{fig:partially_wrong}
\end{figure}

Training the model to assign two different labels for the droplets instead of one allows the criterium of droplets having a closed border around a center region to be used not only during training, but also after segmentation.
To check if a border area such as the ones described in Figure~\ref{sec:algorithm} fullfills this criterium, the algorithm calculates its \emph{alpha shape} and then checks if any of the pixels inside the shape are labeled as \emph{droplet inside}.

The alpha shape of a set of points is a generalization of their \emph{convex hull} and for a real number $\alpha$ includes all edges between the points for which a \emph{generalized disk} with radius $\frac{1}{\alpha}$ can be drawn so that the points of the edge lie on its border and the disk contains no other points. For $\alpha=0$, the generalized disk becomes a half-plane and the alpha shape is equivalent to the convex hull. In the code, the alpha shape is calculated using the \mintinline{python}{alphashape()} function from the \mintinline{python}{alphashape} package, which first calculates the \emph{Delaunay triangulation} of the points and then uses the criterium described above to determine which edges to include in the shape. The parameter $\alpha = 1$ is used since the pixel coordinates are discrete and the pixels are directly adjacent to each other. 

This method of filtering works very well for any but the most extreme cases and improves the accuracy of the measuring process significantly.

\paragraph{The model labels an out of focus droplet or other noise completely with border and center labels.}

This is a much rarer error than the one described in the previous paragraph, but it can still happen. 
It is mostly encountered when the image is very dark overall, which ironically causes the normalization step described in \ref{sec:algorithm} to produce bright artifacts in such images, instead of removing dark spots. 
One example such an image can be seen in Figure~\ref{fig:totally_wrong_a}.
Since identifies in focus droplets by their bright center, this behaviour is not completely unexpected.

However, very rarely the model will assign a false positive even to an area which is not significantly brighter than the background, as can be seen in Figure~\ref{fig:totally_wrong_b}.

\begin{figure}[htbp]
    \centering
    \begin{subfigure}{\textwidth}
        \makebox[\textwidth][c]{
            \begin{tabular}{ll}
                \includegraphics[width=0.45\textwidth]{images/bad4_pic.png}
                &
                \includegraphics[width=0.45\textwidth]{images/bad4_mask.png}
            \end{tabular}
        }
        \caption{The model identifying bright artifacts in dark images as droplets. Note that only bottom left and top right labels are fully closed shapes, which can't be filtered out.}
        \label{fig:totally_wrong_a}
    \end{subfigure}
    \begin{subfigure}{\textwidth}
        \makebox[\textwidth][c]{
            \begin{tabular}{ll}
                \includegraphics[width=0.45\textwidth]{images/bad3_pic.png}
                &
                \includegraphics[width=0.45\textwidth]{images/bad3_mask.png}
            \end{tabular}
        }
        \caption{The model indentifying a comparatively brighter spot as a droplet.}
        \label{fig:totally_wrong_b}
    \end{subfigure}
    \vspace{0.2cm}
    \caption{Examples for cases where the model confidently labels an out of focus droplet or other noise completely with border and center labels. The left image shows the original image, the right image shows the segmentation mask. Blue pixels represent \emph{droplet border} labels, and pink pixels represent \emph{droplet inside} labels.}
    \label{fig:totally_wrong}
\end{figure}

This kind of error is much more severe, since there aren't any criteria to differentiate these kinds of false positives from the structure of the actual droplet labels.
Thus, the only way to mitigate this that comes to mind are improving image preprocessing and/or training the model differently to not make these kinds of mistakes.

One idea to improve the preprocessing is to use a different technique for removing the artifacts. 
The current method of removing the artifacts by subtracting the background image from the original image is very simple and works well if the images in the batch are similar in brightness, but it is not very robust to outliers.
A different approach then, would be to use the background image as a mask for identifying artifact locations instead, and overwrite the pixels in the original image with a more appropriate value. 
This could either be the mean brightness of each individual image, or the values at the same pixel location in the original image with a gaussian blur applied to it.
A process like this could help in minimizing bright artifacts, which seem to be the main cause of this kind of error.

To instead make the model more robust to these kinds of artifacts, some kind of data augmentation could be used that mimics the effect of the artifacts, such as \emph{salt-and-pepper/binary noise}. Another option would be to explicitly use additional samples of images the model seems to find difficult to classify correctly, such as the ones shown in Figure~\ref{fig:totally_wrong}. These examples would need to be labeled of course, but could potentially improve the model substantially in this regard.

\paragraph{In focus criteria may be unrepresentative of actual in focus droplets.} Instead of an error observed in the measuring process, this is a potential problem with the method itself.
Which droplets are considered in focus and should be included in the measurement is a key aspect of the process, since data has to be labeled accordingly. Structures with extremely sharp corners are exceedingly rare in the data captured by the experimental setup, which is why \Citeauthor{kapplAkustischInduzierteVernebelung2022}\cite{kapplAkustischInduzierteVernebelung2022} argues that structures with a dark border and a bright center should be considered sufficiently in focus.

The reasoning behind this is that light that passes through the outer areas of the droplet is scattered more strongly, resulting in less light reaching the camera, while light that passes through the center of the droplet passes through straight, with beams near the center even being focused slightly.

The success of the method as a measurement technique depends heavily on the veracity of this assumption. 
However, if in the future, other criteriums are found to be more suitable to differentiate between in focus and out of focus droplets, the method could theoretically be adapted to use those instead.
    \section{Experimental Setup}
\label{sec:setup}

Athough the detection techniques described in \ref{sec:detection} are in principle adaptable to similar data, the image data used in this thesis is captured using a specific setup. 

The setup uses the \emph{Fastcam Mini UX} in combination with a \emph{Olympus LMPlan IR 50X / 0.55} microscope objective lens as well as a seperate focussing lens to capture images of the droplets.
Together with a parallel light source, this setup forms a microscope with a large magnification factor, which is necessary to capture the droplets in sufficient detail considering their expected size of \SIrange{1}{30}{\micro\meter} \cite{kapplAkustischInduzierteVernebelung2022}.

The SAW device together with a liquid reservoir is placed next to the light source and the camera so that the droplet droplet stream is illuminated from the side. An image of the setup is shown in \ref{fig:setup}.

Images are taken at a resolution of \qtyproduct{1280 x 1024}{\pixel} and a frame rate of \SI{50}{\hertz}. The framerate is kept low to enable capturing data over a longer period of time without saturating the memory of the camera, since the droplets are not constantly in the field of view of the microscope. 

\begin{figure}[htbp]
    \centering
    \makebox[\textwidth][c]{\includegraphics[width=0.9\textwidth]{images/0c35c5ed-b77b-4d8d-b30f-b23403bd04d0.JPG}}
    \vspace{0.2cm}
    \caption{Image of the measurement setup. From left to right the high speed camera, the focusing lens and objective lens, the SAW device and the liquid reservoir as well as the light source are visible.}
    \label{fig:setup}
\end{figure}

    \chapter{Experiments and Results}
    \label{sec:experiments}
    Several techniques are explored during model training in the hopes of improving the detection detection and measurement performance. 
This section evaluates the results of these experiments and discusses their effectiveness.

All learning is done in the \emph{pytorch} frame


    \input{bibliography.tex}

\end{document}